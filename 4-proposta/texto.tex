
\chapter{Teste Wald em modelos multivariados de covariância linear generalizada}

Este capítulo é dedicado à apresentação de nossa proposta: o uso do teste Wald para avaliação dos parâmetros de McGLMs. Vale lembrar que nos McGLMs existem parâmetros de regressão, dispersão e potência e que nossa proposta pode ser aplicada a qualquer parâmetro ou combinação de parâmetros. Além disso, cada conjunto de parâmetros possui uma interpretação prática bastante relevante de tal modo que por meio dos parâmetros de regresão é possível identificar as explicativas relevantes, por meio dos parâmetros de dispersão é possível avaliar o impacto da correlação entre unidades do conjunto de dados e por meio dos parâmetros de potência é possível identificar qual distribuição de probabilidade melhor se adequa ao problema de acordo com a função de variância.

%=====================================================

\section{Hipóteses e estatística de teste}
\subsection{Exemplo 1: hipótese para um único parâmetro}
\subsection{Exemplo 2: hipótese para múltiplos parâmetros}
\subsection{Exemplo 3: hipótese de igualdade de parâmetros}
\subsection{Exemplo 4: hipótese sobre parâmetros de regressão ou dispersão para respostas sob mesmo preditor}

%=====================================================

\section{ANOVA e MANOVA via teste Wald}
\subsection{ANOVA e MANOVA tipo I}
\subsection{ANOVA e MANOVA tipo II}
\subsection{ANOVA e MANOVA tipo III}

%=====================================================

\section{Teste de comparações múltiplas via teste Wald}

%=====================================================