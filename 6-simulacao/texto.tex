\chapter{Estudo de simulação}

%=====================================================

\textbf{TODO}

\begin{itemize}
  
  \item \textbf{Definir estudo de simulação}
  
  \item \textbf{Avaliar desempenho do erro do tipo I, simular por exemplo beta1=1, testar se beta1=1 e ver quantas vezes o teste confirma este resultado. }
  
  \item \textbf{Testar poder simulando valores e testando valores distantes de 0; distância de mahalanobis pode ser uma distância interessante para este fim (não sei se entendi mt bem a ideia).}
  
  \item \textbf{Precisamos responder se o teste é confiável, para que tamanho de amostra ele controla adequadamente o erro do tipo 1, ou temos um teste bom já com pequenas amostras}
  
  \item \textbf{simular de normais matriz variada pois é possível fixar sigma r e b}
  
  \item \textbf{}
  
\end{itemize}


%=====================================================
